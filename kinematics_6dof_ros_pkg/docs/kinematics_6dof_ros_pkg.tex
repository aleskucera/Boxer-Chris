\documentclass[twoside]{article}
\usepackage[a4paper]{geometry}
\geometry{verbose,tmargin=2.5cm,bmargin=2cm,lmargin=2cm,rmargin=2cm}
\usepackage{fancyhdr}
\pagestyle{fancy}

\usepackage{multicol}

% nastavení pisma a češtiny
\usepackage{lmodern}
\usepackage[T1]{fontenc}
\usepackage[cp1250]{inputenc}
%\usepackage[czech]{babel}
\usepackage[british]{babel}	
\usepackage{amsmath}
\usepackage{float}
% odkazy
\usepackage{url}

% vícesloupcové tabulky
\usepackage{multirow}

% vnořené popisky obrázků
\usepackage{subcaption}

% automatická konverze EPS 
\usepackage{graphicx} 
\usepackage{epstopdf}

% odkazy a záložky
\usepackage[unicode=true, bookmarks=true,bookmarksnumbered=true,
bookmarksopen=false, breaklinks=false,pdfborder={0 0 0},
pdfpagemode=UseNone,backref=false,colorlinks=true] {hyperref}

% Poznámky při překladu
\usepackage{xkeyval}	% Inline todonotes
\presetkeys{todonotes}{inline}{}
\usepackage[textsize = footnotesize]{todonotes}

% Zacni sekci slovem ukol
\renewcommand{\thesection}{\arabic{section}}
% enumerate zacina s pismenem
%\renewcommand{\theenumi}{\alph{enumi}}

\renewcommand{\labelenumi}{\arabic{enumi}.}
\renewcommand{\labelenumii}{\alph{enumii}.}
% smaz aktualni page layout
\fancyhf{}
% zahlavi
\usepackage{titling}
\fancyhf[HC]{\thetitle}
\fancyhf[HLE,HRO]{\theauthor}
\fancyhf[HRE,HLO]{\today}
%zapati
\fancyhf[FLE,FRO]{\thepage}

%subfiles package
\usepackage{subfiles}

% údaje o autorovi
\title{Kinematics solver for serial 6-DOF manipulators (ROS1 version)}
\author{Tomas Chaloupecky,\\ Tereza Uhrova}
\date{\today}



\begin{document}

\maketitle
\newpage
%{
%	\hypersetup{linkcolor=black}
%	\tableofcontents
%}
\newpage
\section{Introduction}
This manual describes ROS package solving forward and inverse kinematics of serial 6-DOF manipulator.
Package provides complete solution of 6-DOF kinematics and it's integration into ROS. 

\section{Estimated Kinematics Model}
This solver uses kinematics model \ref{fig:robotschemedh} for computation.
Red circles represent rotation joints and their normal vectors correspond to rotation axis of the joint.
Joint directions are shown with arrows. 
The zero position of each joint is shown by thin black line and zero symbol.

\begin{figure}[H]
	\centering
	\includegraphics[width=0.6\linewidth]{robot_schemeDH}
	\caption{Used kinematics model}
	\label{fig:robotschemedh}
\end{figure}

The solver uses rotation representation by Euler's angles with axis order $ ZXZ $.\\
As your robot is very likely different, please follow the guide in chapter \ref{sec:configuration}.

%\section{Software}
%Software is in form of a ROS1 package.
%The solver it self is called \texttt{kinematics\_6DOF.py} and is located in \texttt{scripts/kinematics\_6DOF} directory.
%This script contains a class called \texttt{KIN\_6DOF}.
%Class provides methods for solving the forward and inverse kinematics.\\
%
%If the package is compiled, the class is accessible from any package, thanks to the file \texttt{setup.py}.\\
\newpage
\section{Configuration}
\label{sec:configuration}
Solver needs for it's correct function following parameters.
These are set in file \texttt{kinematics.yaml} in \text{config/} directory.
\begin{itemize}
	\item \texttt{kinematics\_solver} - kinematics plugin to be used
	\item \texttt{eef\_transformation} - rigid body transformation from 6th frame of model \ref{fig:robotschemedh} to end effector. Transformation is set by six parameters (if Euler angles $ ZXZ $ are used) or seven if quaternions are used.\\
	The format is
	\begin{multicols}{2}
		Euler angles:
		\begin{itemize}
			\item X translation
			\item Y translation
			\item Z translation
			\item Z Euler angle
			\item X Euler angle
			\item Z Euler angle
		\end{itemize}
	\columnbreak
	quaternion:
	\begin{itemize}
		\item X translation
		\item Y translation
		\item Z translation
		\item X quaternion
		\item Y quaternion
		\item Z quaternion
		\item W quaternion
	\end{itemize}
	\end{multicols}

	\item \texttt{base\_transformation} - rigid body transformation from world frame to frame $ S $ of model \ref{fig:robotschemedh}. The format is same as above.
	\item \texttt{link\_lengths} - list of link lengths in order $ l_{1},l_{2},..., l_{6} $.
	\item \texttt{joint\_offset} - zero position in relation to model \ref{fig:robotschemedh} in order $ J1_{offset}, J2_{offset},...,J6_{offset} $.
	\item \texttt{joint\_directions} - list of symbols. If joint positive direction of n-th joint is same, the n-th element of list is '+' and '-' otherwise.
	\item \texttt{configuration} - string of 3 bits. Defines required pose.\todo{konkretni konfigurace}
	\item \texttt{joint\_limits\_names} - list of joint names which are used in system. These are used to download joint limits.
\end{itemize}

\textbf{
	Important: By default file \texttt{kinematics.yaml} sets the the \texttt{SRVKinematicsPlugin} to be used!
	Remove the kinematics settings from your MoveIt configuration if you use it!
}

\section{Launching}
\label{sec:launching}


\end{document}
